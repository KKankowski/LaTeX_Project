\documentclass{article}
\usepackage[T1]{fontenc}
\usepackage[polish]{babel}
\usepackage[utf8]{inputenc}
\usepackage{lmodern}
\selectlanguage{polish}
\usepackage{graphicx}
\graphicspath{ {./images/} }
\usepackage{hyperref}
\usepackage{ulem}
\usepackage{indentfirst}
\usepackage{amsmath}
\usepackage{caption}
\usepackage{wasysym}
\usepackage{sectsty}
\usepackage{multicol}
\sectionfont{\clearpage}

\DeclareCaptionType{equ}[][]

\hypersetup{
    colorlinks=true,
    linkcolor=blue,
    filecolor=magenta,      
    urlcolor=cyan,
}

\begin{document}
	\begin{titlepage}	
		\title{Wind is Coming}
		\author{Kasper Kankowski}
		\date{17.12.2019}
		\maketitle
		\begin{figure}[h]
			\includegraphics[width=\textwidth]{titleImg}
			\caption{Przykład nieznanego kitesurfera pływającego na nieznanym spocie}
		\end{figure}
	\end{titlepage}
	
	
	\tableofcontents

	\newpage	
	\section{Wstęp!}
	\label{sec:wstep}
		Czym tak właściwie jest praca o tajemniczym tytule "Wind is Coming"? Jest to praca o najwspanialszym \sout{(oczywiście zaraz po spaniu na czas i wyjadaniu Nutelli ze słoika)} sporcie na świecie! Ale może dość słodzenia, przejdzmy do konkretów!
		\newline
	
		\begin{figure}[h]
			\centering
			\includegraphics[scale=0.21]{nutella}
			\caption{Przykład nutelli którą możemy wyjadać ze słoika}
		\end{figure}
	
	\subsection{ALE O CZYM MOWA?}
		No właśnie, tytuł naszej ,,subsekcji'' daje do myślenia. O czym mowa? Mniej więcej o tym co akurat przyjdzie mi do głowy tj: wszystko i nic. A właściwie to o Kitesurfingu, który jest sportem extremalnym, dlatego może nie będę się za bardzo rozwijał bo się ludziom spodoba i jeszcze sobie krzywdę zrobią\dots
	
	
	\subsection{Podwstęp}
		Tak by wyszło właśnie\dots \space że Kiteboarding to taki \uline{podwstęp}. Sam jestem tego najlepszym przykładem, wysłany jako 10 latek na kurs. To nie to co myślicie, rodzice po prostu mieli mnie dość i chcieli odpocząć, więc postanowili \textit{się mnie pozbyć} na jakiś czas. No i się im to zaje\dots fajnie udało. Niczego nieświadomy, niespełna sięgający ponad stół Kasper zrozumiał czym jest ADRENALINA. Tak zaczęła się moja jak do tej pory 12 letnia przygoda z wodą, wiatrem, alkoholem i imprezami. Ale to chyba nie ten rozdział, więc zapraszam cię mój drogi czytelniku do czytania dalej. W przypadku gdy nie zrozumiałeś co miałem na myśli, cofnij się proszę do \hyperref[sec:wstep]{wstępu}.
	
	\section{Historia}
	
		\begin{figure}[h]
			\centering
			\includegraphics[width=\textwidth]{tvpHistoria}
			\caption{Chyba nie do końca o to chodziło z historią, ale był to pierwszy obrazek po wpisaniu ,,Historia'' w wujka googla. (Chyba powinienem też pisać krótsze opisy moich obrazków, bo robi się to okrutnie nieczytelne na moim dokumencie napisanym w \LaTeX)}
		\end{figure}
	
	\subsection{Początek}
		Jak już wcześniej wspomniałem. Zapisany na kurs aby zapewnić spokój rodzicom. Nie trwało to jednak długo, bo aż 4 godziny. Z tego co pamiętam były to 2 sesje po 2h, nie jestem na 100\% pewny, bo bałem się zapytać mojej pani instruktor ile czasu byłem na wodzie, więc są to tylko domysły 10 latka. Na szczęście wakacje się kończyły i zdążyłem wrócić do szkoły zanim spróbowałem kite jeszcze raz! Po \sout{cholernie} bardzo ciężkim roku szkolnym, w którym męczyłem się z przyrką i majzą nadszedł czas kolejnych wakacji. Cieszyłem się jak głupi do sera! Nie no spokojnie nie na myśl o kajcie, tylko o tym, że nie będę musiał przez 2 miesiące oglądać tego aresztu dla nieletnich. Finalnie jak się domyślasz znów wylądowałem na wodzie, a właściwie to pod nią i to nie raz. \cite{wiedza_wlasna}
	
	\subsection{Pierwsze ślizgi}
	
		Już już spokojnie! Udało się! popłynąłem! To był wielki krok dla mnie a jeszcze większy dla moich rodziców?! Od teraz dawałem im spokój na jeszcze dłużej, bo nie byłem w stanie pływać tak, aby wracać do miejsca w którym zaczynałem. Wiązało się to z tym, że idąc pływać - wracałem po jakiś 5 godzinach. Około 1,5h pływania a reszta czasu na marsz wzdłuż półwyspu helskiego prosto do przyczepy kempingowej dziadka. Z czasem zaczęło to wyglądać coraz lepiej. Chociaż nie ukrywam, że zanim doszedłem do etapu samodzielności i radzenia sobie w 80\% z przypadkami które mi się zdarzały, to zebrałem tyle glonów zębami, że aż polubiłem shushi. 
	
	\subsection{Pierwszy alkohol}	
			
		Oczywiście, że tak! Mogłeś się tego domyślić mój drogi czytelniku. Życie surfera to nie tylko sport, sport, sport, życie surfera to także imprezy i alkohol. W moim przypadku było inaczej, raczej coś jak: Dwóch 16 latków, jeden redd's, nieskończona żenada. No przynajmniej z tego co pamiętam to chociaż smaczny był. Zawsze mogłem trafić na jakiegoś taniego, ciepłego browara w cenie 1,65 za puszkę 0,5l. No ale jakby nie patrzeć to zawsze grzeczny byłem, więc nigdy więcej się takie incydenty nie zdarzały.
		
		\begin{figure}[h]
			\centering
			\includegraphics[]{redds}
			\caption{Sprawca całego zamieszania, pamiętny redds}
		\end{figure}
		
	\subsection{Pierwsza praca}
	
	Właściwie to nie wiem który podrozdział powinien pojawić się pierwszy alkohol czy praca. Mowa oczywiście o chronologii! Przyjmijmy więc, że epizody te są umieszczone równolegle na linii czasu i dzieją się niezależnie, ale nadal równolegle. Pracę załatwiła mi mama (no o ile robienie najnudniejszej, najżmudniejszej, najgorszej i niechcianej roboty za darmo można nazwać pracą). Ale nie ma co narzekać po jednych wakacjach jako niewolnik awansowałem na dumnie brzmiące stanowisko \textbf{\uline{,,Młodszego Instruktora''}}. Nadal robiłem czarną robotę i dostawałem pieniądze wg wzoru: 
	
	\begin{eqnarray}
		\frac{najnizsza.krajowa}{wymagania.szefa}*\int_{poczatek.wakacji}^{koniec.wakacji}{krew*pot*lzy \; \; \mathrm{d} krew}
	\end{eqnarray}
	
	Ale! Ale byłem instruktorem, uczyłem ludzi tego co sam kocham i dostawałem za to pieniądze! Przy okazji miałem możliwośc pływania na \sout{starych, zniszczonych latawcach na których uczyli się moi kursanci}! No okej miałem swój sprzęt, nie było dramatu.
	
	\subsection{Kolejna Praca}
	Po 2 latach pracy jako pół intruktor, pół dzieciak od czarnej roboty udało mi się znaleźć pracę w innej szkółce. Bunkrów nie było, ale też było fajnie. Pracowałem z ludźmi mającymi ogromne doświadczenie i niesamowicie dużo się od nich nauczyłem (odniosłem też wrażenie, że doświadczenie rośnie wraz z ilością alkoholu, który człowiek jest w stanie przyjąć\dots \space no cóż). W kolejne wakacje szkółkę przęli nowi właściciele, którzy zorganizowali dla nas międzynarodowy kurs instruktorski IKO - Międzynarodowej organizacji kitesurfingu. Podniosłem swoje kwalifikacje i rok później zostałem kierownikiem szkółki.
	
	\begin{figure}[h]
			\centering
			\includegraphics[scale=0.2]{iko}
			\caption{Podobno logo IKO, nawet nasz szkoleniowiec nie był pewny jak wygląda}
		\end{figure}
	
	\subsection{Kolejna praca, większa odpowiedzialność}
	Co tu dużo pisać zostałem osobą odpowiedzialną za instruktorów, poziom szkolenia i sprzęt, zajęty byłem mocno. Jak to w tym \href{https://www.computerworld.pl/felieton/Pusta-taczka,298952.html}{żarcie} opowiadają, nie miałem czasu ładować taczki. Bardzo dużo zajmowałem się tabelkami takimi jak tabela nr: \ref{table:tablica} lub \ref{table:tablica2}
	
	
	\begin{table}[]
		\centering
		\begin{tabular}{llll}
Data    & Kursant          & Kurs         & Czas  \\
6/19/17 & Aga              & IKO2 ind     & 2     \\
6/24/17 & Basia, Jagoda    & IKO 3? ind   & 2     \\
7/5/17  & Dorota           & IKO 1 ind    & 2     \\
7/7/17  & Paweł, Agnieszka & IKO 1/2 ind  & 1.5   \\
7/9/17  & Filip            & IKO 2 ind    & 1.5   \\
7/9/17  & Dobromir         & IKO 2/3 ind  & 1.5   \\
7/11/17 & Emma             & IKO 1 ind    & 3.5   \\
7/13/17 & Patryk           & IKO 3 ind    & 2     \\
7/15/17 & Paweł            & IKO 1 ind    & 2     \\
7/16/17 & Paweł            & IKO 1 ind    & 2     \\
7/17/17 & Dorian           & IKO 4 ind    & 1.5   \\
7/22/17 & Kuba, Darek      & IKO 1 ind    & 3     \\
6/27/17 & Kuba, Darek      & IKO 1 ind    & 2     \\
8/2/17  & fabian           & iko 2/3 ind  & 2     \\
8/4/17  & Fabian           & Iko 2/3 ind  & 2     \\
8/5/17  & Michał           & IKO 3 ind    & 4     \\
8/7/17  & Sara             & Iko 3 ind    & 2     \\
8/7/17  & mieszko          & iko 1 ind    & 1     \\
8/8/17  & Leszek           & Iko 1 ind    & 2     \\
8/8/17  & Jacek            & iko 3 ind    & 2     \\
8/9/17  & leszek           & iko 2 ind    & 1.5   \\
8/15/17 & Daria            & iko 3 ind    & 3.5   \\
8/15/17 & michał           & iko 3 ind    & 2     
		\end{tabular}
		
	\caption{Tabela sam nie wiem czego, kiedyś pamiętałem}
	\label{table:tablica}
	\end{table}
	
	\begin{table}[]
	\centering
	\begin{tabular}{lllll}
	data       & czas & woda/wyciąg \\
	24-07-2017 & 3    & Ląd         \\
	25-07-2017 & 2    & Woda        \\
	26-07-2017 & 1    & Woda        \\
	03-07-2017 & 1    & Ląd         \\
	23-07-2017 & 1    & Ląd         \\
	29-07-2017 & 0,5  & Ląd         \\
	30-07-2017 & 1    & Woda        \\
	17-08-2017 & 0,5  & Ponton            
\end{tabular}
\caption{Tabela obsługi wake jednego z instruktorów}
\label{table:tablica2} 
\end{table}
	
	
	na moje szczęśćie ominęły mnie wzory takie jak te: 
	
	\begin{eqnarray}
			f(x) &= x^2\\
  			g(x) &= \frac{1}{x}\\
  			F(x) &= \int^a_b \frac{1}{3}x^3
	\end{eqnarray}
	
	
	

	\section{CheckLista}
	
	\begin{itemize}
	\item \makebox[1.5cm]{}Przedstawić o czym będzie praca \CheckedBox
	\item \makebox[1.5cm]{}Pogadać o nutelli \CheckedBox
	\item \makebox[1.5cm]{}Opowiedzieć historię swojego życia \CheckedBox
	\item \makebox[1.5cm]{}Użyć trochę niezrozumiałych kiteboardingowych słów \CheckedBox
	\item \makebox[1.5cm]{}Wyłumaczyć czytelnikowi czym jest Kitesurfing\dots \space Oh\dots \space wait\dots
	\end{itemize}
	
	\section{Czym Jest Kitesurfing?}
	Kitesurfing (także kiteboarding) – sport wodny, polegający na poruszaniu się po wodzie na desce lub hydroskrzydle (foil) z pomocą pędnika - latawca (kite).

Sport ten uprawiany jest zarówno na akwenach słodkowodnych (sprzyjające wiatry wieją m.in. w dolinach alpejskich, np. w Południowym Tyrolu), jak i morskich (np. na polskim wybrzeżu Bałtyku).

Od igrzysk letnich w Rio de Janeiro w 2016 roku kitesurfing miał dołączyć do konkurencji żeglarskich, zastępując tym samym windsurfing. Jednak decyzją światowej federacji żeglarskiej ISAF dnia 10.11.2012 r. został skreślony z listy konkurencji na olimpiadzie w Rio na rzecz RS:X. \cite{kitesurf_wiki}

\begin{multicols}{2}
[
\section{Czy Kiteboarding może być bezpieczny?}
Bezpieczeństwo jest odwrotnością zagrożenia. Postrzegane jest ono jako stan, który charakteryzuje się niskim
poziomem ryzyka utraty czegoś szczególnie cennego – życia, zdrowia, pracy, szacunku, uczuć, dóbr materialnych lub
niematerialnych, itp. Dając człowiekowi poczucie pewności jest jedną z podstawowych potrzeb w piramidzie Maslowa.
Niniejsza praca stanowi fragment obszernych badań związanych z bezpieczeństwem w różnych obszarach kultury fizycznej.
Problematyka bezpieczeństwa ma szczególną rangę w takich formach aktywności, w których zagrożenia dla życia i zdrowia
uczestników jest bardzo duże. 
]
Jedną z takich form, jest kitesurfing, który jako dyscyplina sportu został właśnie włączony do programu Igrzysk
Olimpijskich w Paryżu 2024. „Mimo, iż pierwsze doniesienia o wykorzystaniu latawców do napędu łódek sięgają aż kilkuset
lat przed naszą erą, to jednak dopiero na początku XIX wieku chęć wykorzystania latawca nabrała dynamiki. Historia
kitesurfingu jest stosunkowo krótka, ale bardzo bogata” [2].
Kitesurfingu nie można dziś pomijać, mówiąc o sportach wodnych. O ile jeszcze kilka lat temu, latawce pojawiały się
na kilku akwenach, o tyle dziś, w wietrzny dzień, stanowią pokaźny procent użytkowników każdego spotu.
O jego popularności decyduje wiele czynników. Podobnie jak niegdyś np. snowboard, jest bardzo widowiskowy –
zaliczany do grupy sportów ekstremalnych (choć dzięki postępom w konstrukcji sprzętu jest dziś bardzo bezpieczny), dla
osób oglądających oraz pływających kitesurferów dostarcza wielu emocji. Oprócz tego, jest tą formą rekreacji, którą można
bardzo szybko opanować - średnio na naukę do poziomu pływania w ślizgu potrzeba 2 dni nauki.
„Specyfika kitesurfingu polega również na mobilności – latawiec wraz z osprzętem można łatwo spakować do
plecaka, deskę chwycić dłoń i bez problemu się przemieszczać, co nie jest takie proste w przypadku wielu innych sportów
wodnych”[3].
Dodatkowo wprowadza on trzeci wymiar a więc możliwość wzbijania się w powietrze i pozostawania nad wodą
czasami przez kilkanaście sekund.
Rynek kitesurfingowy w Polsce rozwinął się bardzo mocno na przestrzeni ostatnich 10 lat.
Nad polskim morzem powstało wiele szkół nauczających kitesurfingu - w lecie 2018 roku było ich ponad 60.
Według danych szacunkowych, w ostatnim sezonie letnim w każdej ze szkół nauczyło się pływać średnio 200 osób, co
w podsumowaniu tylko 3 ostatnich lat liczbę nowych osób (uczących się – wchodzących do tego sportu) w kitesurfingu
w Polsce określa na około 35000. Należy wziąć pod uwagę fakt, iż część osób uczy się kitesurfingu sama, część zaś korzysta
z kursów podczas wypraw do ciepłych krajów.
Zainteresowanie kitesurfingiem rośnie bardzo mocno z roku na rok, co oznacza, że w kolejnych latach liczba
zainteresowanych nauką będzie coraz większa.
O wzroście popularności tego sportu świadczyć mogą następujące fakty, iż poza granicami Polski powstały już
polskie bazy-szkoły kitesurfingowe: Tarifa, El Gouna (Egipt), Rodos, Paros (Grecja) a wielu Polaków naucza kitesurfingu m.in.
na Cabarete, Cape Town. ~\cite{wiesner2018zarzkadzanie}
\end{multicols}
	
	\newpage
	\addcontentsline{toc}{section}{\listfigurename}
	\listoffigures
	\addcontentsline{toc}{section}{\listtablename}
	\listoftables
	
	\newpage
	\bibliographystyle{plain}
	\bibliography{b}
	
	
	
\end{document}