\documentclass{article}
\usepackage[T1]{fontenc}
\usepackage[polish]{babel}
\usepackage[utf8]{inputenc}
\usepackage{lmodern}
\selectlanguage{polish}
\usepackage{graphicx}
\graphicspath{ {./images/} }
\usepackage{hyperref}
\usepackage{ulem}
\usepackage{indentfirst}

\hypersetup{
    colorlinks=true,
    linkcolor=blue,
    filecolor=magenta,      
    urlcolor=cyan,
}

\begin{document}
	\begin{titlepage}	
		\title{Wind is Coming}
		\author{Kasper Kankowski}
		\date{17.12.2019}
		\maketitle
		\begin{figure}[h]
			\includegraphics[width=\textwidth]{titleImg}
			\caption{Przykład nieznanego kitesurfera pływającego na nieznanym spocie}
		\end{figure}
	\end{titlepage}
	
	
	\tableofcontents

	\newpage	
	\section{Wstęp!}
	\label{sec:wstep}
		Czym tak właściwie jest praca o tajemniczym tytule "Wind is Coming"? Jest to praca 		o najwspanialszym \sout{(oczywiście zaraz po spaniu na czas i wyjadaniu Nutelli ze 		słoika)} sporcie na świecie! Ale może dość słodzenia, przejdzmy do konkretów!
		\newline
	
		\begin{figure}[h]
			\centering
			\includegraphics[scale=0.21]{nutella}
			\caption{Przykład nutelli którą możemy wyjadać ze słoika}
		\end{figure}
	
	\subsection{ALE O CZYM MOWA?}
		No właśnie, tytuł naszej ,,subsekcji'' daje do myślenia. O czym mowa? Mniej więcej 		o tym co akurat przyjdzie mi do głowy tj: wszystko i nic. A właściwie to o 					Kitesurfingu, który jest sportem extremalnym, dlatego może nie będę się za bardzo 			rozwijał bo się ludziom spodoba i jeszcze sobie krzywdę zrobią\dots
	
	
	\subsection{Podwstęp}
		Tak by wyszło właśnie\dots \space że Kiteboarding to taki \uline{podwstęp}. Sam 			jestem tego najlepszym przykładem, wysłany jako 10 latek na kurs. To nie to co 				myślicie, rodzice po prostu mieli mnie dość i chcieli odpocząć, więc postanowili 			\textit{się mnie pozbyć} na jakiś czas. No i się im to zaje\dots fajnie udało. 				Niczego nieświadomy, niespełna sięgający ponad stół Kasper zrozumiał czym jest 				ADRENALINA. Tak zaczęła się moja jak do tej pory 12 letnia przygoda z wodą, 				wiatrem, alkoholem i imprezami. Ale to chyba nie ten rozdział, więc zapraszam cię 			mój drogi czytelniku do czytania dalej. W przypadku gdy nie zrozumiałeś co miałem 			na myśli, cofnij się proszę do \hyperref[sec:wstep]{wstępu}.
		
	\section{Historia}
	
	\begin{figure}[h]
			\centering
			\includegraphics[width=\textwidth]{tvpHistoria}
			\caption{Chyba nie do końca o to chodziło z historią, ale był to pierwszy 					obrazek po wpisaniu ,,Historia'' w wujka googla. (Chyba powinienem też pisać 				krótsze opisy moich obrazków, co uważasz czytelniku?}
		\end{figure}
	
	\subsection{Początek}
		Jak już wcześniej wspomniałem. Zapisany na kurs aby zapewnić spokój rodzicom. Nie 			trwało to jednak długo, bo aż 4 godziny. Z tego co pamiętam były to 2 sesje po 2h, 		nie jestem na 100\% pewny, bo bałem się zapytać mojej pani instruktor ile czasu 			byłem na wodzie, więc są to tylko domysły 10 latka. Na szczęście wakacje się 				kończyły i zdążyłem wrócić do szkoły zanim spróbowałem kite jeszcze raz! Po 				\sout{cholernie} bardzo ciężkim roku szkolnym, w którym męczyłem się z przyrką i 			majzą nadszedł czas kolejnych wakacji. Cieszyłem się jak głupi do sera! Nie no 				spokojnie nie na myśl o kajcie, tylko o tym, że nie będę musiał przez 2 miesiące 			oglądać tego aresztu dla nieletnich. Finalnie jak się domyślasz znów wylądowałem 			na wodzie, a właściwie to pod nią i to nie raz.
	
	\subsection{Pierwsze ślizgi}
	
		Już już spokojnie! Udało się! popłynąłem! To był wielki krok dla mnie a jeszcze 			większy dla moich rodziców?! Od teraz dawałem im spokój na jeszcze dłużej, bo nie 			byłem w stanie pływać tak, aby wracać do miejsca w którym zaczynałem. Wiązało się 			to z tym, że idąc pływać - wracałem po jakiś 5 godzinach. Około 1,5h pływania a 			reszta czasu na marsz wzdłuż półwyspu helskiego prosto do przyczepy kempingowej 			dziadka.  
	
	\subsection{Pierwszy alkohol}	
	
	\subsection{Pierwsza praca}
	
	\subsection{Pierwsze narkotyki}
	
	\newpage
	\addcontentsline{toc}{section}{\listfigurename}
	\listoffigures
	\newpage
	\addcontentsline{toc}{section}{\listtablename}
	\listoftables
	\newpage

	
	\begin{thebibliography}{9}
		\bibitem{overleaf.com} 
		\href{https://www.overleaf.com/learn/latex/Inserting_Images}{overleaf.com 			Inserting Images}, \href{https://www.overleaf.com/learn/latex/Hyperlinks}{overleaf.com 			Hyperlinks} Autor nieznany, 2019.
		
		\bibitem{pakietomat.wordpress.com} 
		\href{https://pakietomat.wordpress.com/tag/przekreslenia/}{Pakietomat przekreślenia}, Zofia Walczak, 2013.
		
		\bibitem{NAJWSPANIALSZY} 
		Wiedza własna nabyta przez lata doświadczeń w kitesurfingu, Kasper 'Kaspi' Kankowski, 2019.

		
	\end{thebibliography} 
\end{document}